\documentclass[]{article}
\usepackage{lmodern}
\usepackage{amssymb,amsmath}
\usepackage{ifxetex,ifluatex}
\usepackage{fixltx2e} % provides \textsubscript
\ifnum 0\ifxetex 1\fi\ifluatex 1\fi=0 % if pdftex
  \usepackage[T1]{fontenc}
  \usepackage[utf8]{inputenc}
\else % if luatex or xelatex
  \ifxetex
    \usepackage{mathspec}
  \else
    \usepackage{fontspec}
  \fi
  \defaultfontfeatures{Ligatures=TeX,Scale=MatchLowercase}
\fi
% use upquote if available, for straight quotes in verbatim environments
\IfFileExists{upquote.sty}{\usepackage{upquote}}{}
% use microtype if available
\IfFileExists{microtype.sty}{%
\usepackage{microtype}
\UseMicrotypeSet[protrusion]{basicmath} % disable protrusion for tt fonts
}{}
\usepackage[margin=1in]{geometry}
\usepackage{hyperref}
\hypersetup{unicode=true,
            pdfborder={0 0 0},
            breaklinks=true}
\urlstyle{same}  % don't use monospace font for urls
\usepackage{color}
\usepackage{fancyvrb}
\newcommand{\VerbBar}{|}
\newcommand{\VERB}{\Verb[commandchars=\\\{\}]}
\DefineVerbatimEnvironment{Highlighting}{Verbatim}{commandchars=\\\{\}}
% Add ',fontsize=\small' for more characters per line
\usepackage{framed}
\definecolor{shadecolor}{RGB}{248,248,248}
\newenvironment{Shaded}{\begin{snugshade}}{\end{snugshade}}
\newcommand{\KeywordTok}[1]{\textcolor[rgb]{0.13,0.29,0.53}{\textbf{#1}}}
\newcommand{\DataTypeTok}[1]{\textcolor[rgb]{0.13,0.29,0.53}{#1}}
\newcommand{\DecValTok}[1]{\textcolor[rgb]{0.00,0.00,0.81}{#1}}
\newcommand{\BaseNTok}[1]{\textcolor[rgb]{0.00,0.00,0.81}{#1}}
\newcommand{\FloatTok}[1]{\textcolor[rgb]{0.00,0.00,0.81}{#1}}
\newcommand{\ConstantTok}[1]{\textcolor[rgb]{0.00,0.00,0.00}{#1}}
\newcommand{\CharTok}[1]{\textcolor[rgb]{0.31,0.60,0.02}{#1}}
\newcommand{\SpecialCharTok}[1]{\textcolor[rgb]{0.00,0.00,0.00}{#1}}
\newcommand{\StringTok}[1]{\textcolor[rgb]{0.31,0.60,0.02}{#1}}
\newcommand{\VerbatimStringTok}[1]{\textcolor[rgb]{0.31,0.60,0.02}{#1}}
\newcommand{\SpecialStringTok}[1]{\textcolor[rgb]{0.31,0.60,0.02}{#1}}
\newcommand{\ImportTok}[1]{#1}
\newcommand{\CommentTok}[1]{\textcolor[rgb]{0.56,0.35,0.01}{\textit{#1}}}
\newcommand{\DocumentationTok}[1]{\textcolor[rgb]{0.56,0.35,0.01}{\textbf{\textit{#1}}}}
\newcommand{\AnnotationTok}[1]{\textcolor[rgb]{0.56,0.35,0.01}{\textbf{\textit{#1}}}}
\newcommand{\CommentVarTok}[1]{\textcolor[rgb]{0.56,0.35,0.01}{\textbf{\textit{#1}}}}
\newcommand{\OtherTok}[1]{\textcolor[rgb]{0.56,0.35,0.01}{#1}}
\newcommand{\FunctionTok}[1]{\textcolor[rgb]{0.00,0.00,0.00}{#1}}
\newcommand{\VariableTok}[1]{\textcolor[rgb]{0.00,0.00,0.00}{#1}}
\newcommand{\ControlFlowTok}[1]{\textcolor[rgb]{0.13,0.29,0.53}{\textbf{#1}}}
\newcommand{\OperatorTok}[1]{\textcolor[rgb]{0.81,0.36,0.00}{\textbf{#1}}}
\newcommand{\BuiltInTok}[1]{#1}
\newcommand{\ExtensionTok}[1]{#1}
\newcommand{\PreprocessorTok}[1]{\textcolor[rgb]{0.56,0.35,0.01}{\textit{#1}}}
\newcommand{\AttributeTok}[1]{\textcolor[rgb]{0.77,0.63,0.00}{#1}}
\newcommand{\RegionMarkerTok}[1]{#1}
\newcommand{\InformationTok}[1]{\textcolor[rgb]{0.56,0.35,0.01}{\textbf{\textit{#1}}}}
\newcommand{\WarningTok}[1]{\textcolor[rgb]{0.56,0.35,0.01}{\textbf{\textit{#1}}}}
\newcommand{\AlertTok}[1]{\textcolor[rgb]{0.94,0.16,0.16}{#1}}
\newcommand{\ErrorTok}[1]{\textcolor[rgb]{0.64,0.00,0.00}{\textbf{#1}}}
\newcommand{\NormalTok}[1]{#1}
\usepackage{graphicx,grffile}
\makeatletter
\def\maxwidth{\ifdim\Gin@nat@width>\linewidth\linewidth\else\Gin@nat@width\fi}
\def\maxheight{\ifdim\Gin@nat@height>\textheight\textheight\else\Gin@nat@height\fi}
\makeatother
% Scale images if necessary, so that they will not overflow the page
% margins by default, and it is still possible to overwrite the defaults
% using explicit options in \includegraphics[width, height, ...]{}
\setkeys{Gin}{width=\maxwidth,height=\maxheight,keepaspectratio}
\IfFileExists{parskip.sty}{%
\usepackage{parskip}
}{% else
\setlength{\parindent}{0pt}
\setlength{\parskip}{6pt plus 2pt minus 1pt}
}
\setlength{\emergencystretch}{3em}  % prevent overfull lines
\providecommand{\tightlist}{%
  \setlength{\itemsep}{0pt}\setlength{\parskip}{0pt}}
\setcounter{secnumdepth}{0}
% Redefines (sub)paragraphs to behave more like sections
\ifx\paragraph\undefined\else
\let\oldparagraph\paragraph
\renewcommand{\paragraph}[1]{\oldparagraph{#1}\mbox{}}
\fi
\ifx\subparagraph\undefined\else
\let\oldsubparagraph\subparagraph
\renewcommand{\subparagraph}[1]{\oldsubparagraph{#1}\mbox{}}
\fi

%%% Use protect on footnotes to avoid problems with footnotes in titles
\let\rmarkdownfootnote\footnote%
\def\footnote{\protect\rmarkdownfootnote}

%%% Change title format to be more compact
\usepackage{titling}

% Create subtitle command for use in maketitle
\newcommand{\subtitle}[1]{
  \posttitle{
    \begin{center}\large#1\end{center}
    }
}

\setlength{\droptitle}{-2em}

  \title{}
    \pretitle{\vspace{\droptitle}}
  \posttitle{}
    \author{}
    \preauthor{}\postauthor{}
    \date{}
    \predate{}\postdate{}
  
\usepackage{float}

\begin{document}

\begin{centering}

\vspace*{5 cm}

\Huge

{\bf Descubrimiento del Conocimiento usando herramientas de Big Data Módulo 2}

\vspace{3 cm}

\Large
Marco Andrés Vázquez Hernández

\vspace{1 cm}
\normalsize
Práctica KNN. 

Septiembre de 2018

\normalsize
Instituto Politécnico Nacional


\end{centering}

\newpage

\section{Descripción}\label{descripcion}

Con las muesras de entrenamiento, clasificar las nuevas instancias.
Suerte

\section{Carga de archivos}\label{carga-de-archivos}

\begin{Shaded}
\begin{Highlighting}[]
\KeywordTok{setwd}\NormalTok{(}\StringTok{"C:/Users/marco/IPN_BigData/Modulo2/Práctica_KNN"}\NormalTok{)}

\NormalTok{train<-}\KeywordTok{read.csv}\NormalTok{(}\StringTok{"Ejercicio Knn.csv"}\NormalTok{, }\DataTypeTok{skip=}\DecValTok{1}\NormalTok{, }\DataTypeTok{nrows=} \DecValTok{12}\NormalTok{)}
\NormalTok{eval<-}\KeywordTok{read.csv}\NormalTok{(}\StringTok{"Ejercicio Knn.csv"}\NormalTok{, }\DataTypeTok{skip=}\DecValTok{16}\NormalTok{, }\DataTypeTok{nrows=}\DecValTok{4}\NormalTok{, }\DataTypeTok{header=}\NormalTok{F)}
\KeywordTok{colnames}\NormalTok{(eval)<-}\KeywordTok{colnames}\NormalTok{(train)}
\end{Highlighting}
\end{Shaded}

\section{Transformación de datos}\label{transformacion-de-datos}

Se convirtió la variable ``Invertir'' a dicotómica y se juntaron los
datos para tomar todos los valores en la normalización.

\begin{Shaded}
\begin{Highlighting}[]
\NormalTok{train}\OperatorTok{$}\NormalTok{Invertir<-}\KeywordTok{ifelse}\NormalTok{(}\KeywordTok{as.character}\NormalTok{(train}\OperatorTok{$}\NormalTok{Invertir)}\OperatorTok{==}\StringTok{"Si"}\NormalTok{,}\DecValTok{1}\NormalTok{,}\DecValTok{0}\NormalTok{)}
\NormalTok{eval}\OperatorTok{$}\NormalTok{Invertir<-}\KeywordTok{ifelse}\NormalTok{(}\KeywordTok{as.character}\NormalTok{(eval}\OperatorTok{$}\NormalTok{Invertir)}\OperatorTok{==}\StringTok{"Si"}\NormalTok{,}\DecValTok{1}\NormalTok{,}\DecValTok{0}\NormalTok{)}
\NormalTok{dats<-}\KeywordTok{rbind}\NormalTok{(train,eval)}
\end{Highlighting}
\end{Shaded}

Se creó la función para normalizar los datos:

\begin{Shaded}
\begin{Highlighting}[]
\NormalTok{Normalizar<-}\StringTok{ }\ControlFlowTok{function}\NormalTok{ (x)\{}
  \ControlFlowTok{if}\NormalTok{ (}\KeywordTok{all}\NormalTok{(}\KeywordTok{is.na}\NormalTok{(x)))\{}
    \KeywordTok{return}\NormalTok{(}\KeywordTok{rep}\NormalTok{(}\OtherTok{NA}\NormalTok{, }\KeywordTok{length}\NormalTok{(x)))}
\NormalTok{  \} }\ControlFlowTok{else} \ControlFlowTok{if}\NormalTok{ (}\KeywordTok{sum}\NormalTok{(x)}\OperatorTok{==}\DecValTok{0}\NormalTok{)\{}
    \KeywordTok{return}\NormalTok{(}\KeywordTok{rep}\NormalTok{(}\DecValTok{0}\NormalTok{,}\KeywordTok{length}\NormalTok{(x)))}
\NormalTok{  \} }\ControlFlowTok{else} \ControlFlowTok{if}\NormalTok{ (}\KeywordTok{min}\NormalTok{(x)}\OperatorTok{==}\KeywordTok{max}\NormalTok{(x) }\OperatorTok{&}\StringTok{ }\KeywordTok{max}\NormalTok{(x)}\OperatorTok{!=}\DecValTok{0}\NormalTok{)\{}
    \KeywordTok{return}\NormalTok{(}\KeywordTok{rep}\NormalTok{(}\DecValTok{1}\NormalTok{,}\KeywordTok{length}\NormalTok{(x)))}
\NormalTok{  \} }\ControlFlowTok{else}
  \KeywordTok{return}\NormalTok{((x}\OperatorTok{-}\KeywordTok{min}\NormalTok{(x))}\OperatorTok{/}\NormalTok{(}\KeywordTok{max}\NormalTok{(x)}\OperatorTok{-}\KeywordTok{min}\NormalTok{(x)))}
\NormalTok{\} }
\end{Highlighting}
\end{Shaded}

Se aplicó a la base con todos los valores y después se separaron de
nuevo el conjunto de entrenamiento y el de evaluación:

\begin{Shaded}
\begin{Highlighting}[]
\NormalTok{aux<-}\KeywordTok{sapply}\NormalTok{(dats[,}\DecValTok{1}\OperatorTok{:}\NormalTok{(}\KeywordTok{ncol}\NormalTok{(dats)}\OperatorTok{-}\DecValTok{1}\NormalTok{)], Normalizar)}
\NormalTok{train2<-}\KeywordTok{as.data.frame}\NormalTok{(}\KeywordTok{cbind}\NormalTok{(aux[}\DecValTok{1}\OperatorTok{:}\KeywordTok{nrow}\NormalTok{(train),], train[,}\KeywordTok{ncol}\NormalTok{(train)]))}
\KeywordTok{colnames}\NormalTok{(train2)[}\KeywordTok{ncol}\NormalTok{(train2)]<-}\StringTok{"Invertir"}
\NormalTok{eval2<-}\KeywordTok{as.data.frame}\NormalTok{(}\KeywordTok{cbind}\NormalTok{(aux[(}\KeywordTok{nrow}\NormalTok{(train)}\OperatorTok{+}\DecValTok{1}\NormalTok{)}\OperatorTok{:}\KeywordTok{nrow}\NormalTok{(aux),], eval[,}\KeywordTok{ncol}\NormalTok{(eval)]))}
\KeywordTok{colnames}\NormalTok{(eval2)[}\KeywordTok{ncol}\NormalTok{(eval2)]<-}\StringTok{"Invertir"}
\end{Highlighting}
\end{Shaded}

Una muestra de los datos normalizados queda:

\begin{Shaded}
\begin{Highlighting}[]
\KeywordTok{head}\NormalTok{(train2)}
\end{Highlighting}
\end{Shaded}

\begin{verbatim}
##       Precio Metros.Cuadrados Baños Cuartos Estacionamiento Mantenimiento
## 1 0.02439024           0.0625     0       0             0.0     0.1052632
## 2 0.07317073           0.3125     0       1             0.0     0.0000000
## 3 0.26829268           0.2500     0       0             0.5     0.4736842
## 4 0.39024390           0.3125     0       0             0.5     0.3684211
## 5 0.09756098           0.2125     0       0             0.0     0.3684211
## 6 0.14634146           0.6875     1       0             0.5     0.5263158
##   Invertir
## 1        1
## 2        0
## 3        1
## 4        1
## 5        0
## 6        0
\end{verbatim}

\begin{Shaded}
\begin{Highlighting}[]
\KeywordTok{head}\NormalTok{(eval2)}
\end{Highlighting}
\end{Shaded}

\begin{verbatim}
##       Precio Metros.Cuadrados Baños Cuartos Estacionamiento Mantenimiento
## 1 0.04878049           0.1875     0       0             0.5     0.2105263
## 2 0.21951220           0.2500     0       0             0.5     0.4736842
## 3 0.34146341           0.3375     0       1             0.5     0.6842105
## 4 0.73170732           0.8125     1       1             0.5     0.8947368
##   Invertir
## 1       NA
## 2       NA
## 3       NA
## 4       NA
\end{verbatim}

\section{Algoritmo}\label{algoritmo}

Se crearon matrices para medir las distancias de los puntos de
evaluación a cada uno de los puntos en el conjunto de entrenamiento:

\begin{Shaded}
\begin{Highlighting}[]
\NormalTok{aux<-}\KeywordTok{data.frame}\NormalTok{()}
\ControlFlowTok{for}\NormalTok{(i }\ControlFlowTok{in} \DecValTok{1}\OperatorTok{:}\KeywordTok{nrow}\NormalTok{(eval2))\{}
  \ControlFlowTok{for}\NormalTok{(j }\ControlFlowTok{in} \DecValTok{1}\OperatorTok{:}\KeywordTok{nrow}\NormalTok{(train2))\{}
\NormalTok{    aux[j,i]<-}\KeywordTok{dist}\NormalTok{(}\KeywordTok{rbind}\NormalTok{(train2[j,}\OperatorTok{-}\KeywordTok{ncol}\NormalTok{(train2)], eval2[i,}\OperatorTok{-}\KeywordTok{ncol}\NormalTok{(eval2)]), }\DataTypeTok{method=}\StringTok{"euclidean"}\NormalTok{)    }
\NormalTok{  \}}
\NormalTok{\}}
\NormalTok{euclidian<-aux}
\NormalTok{euclidian}
\end{Highlighting}
\end{Shaded}

\begin{verbatim}
##           V1         V2        V3        V4
## 1  0.5265930 0.67746801 1.3271550 1.9839522
## 2  1.1447887 1.22462190 0.8892413 1.6535490
## 3  0.3483446 0.04878049 1.0282663 1.6457353
## 4  0.3964253 0.21008545 1.0501083 1.6259168
## 5  0.5271957 0.52664854 1.1936651 1.8136014
## 6  1.1658650 1.09523070 1.4783448 1.2222979
## 7  2.1067139 1.92319253 1.4907380 0.6068093
## 8  1.8110267 1.67441738 1.2358558 0.2638279
## 9  1.3394420 1.17460247 0.4087944 1.0869831
## 10 0.2806603 0.09622996 1.0511393 1.7060374
## 11 0.5760704 0.76489368 1.3959397 2.0606081
## 12 0.1341810 0.34536999 1.1659718 1.8527778
\end{verbatim}

\begin{Shaded}
\begin{Highlighting}[]
\NormalTok{aux<-}\KeywordTok{data.frame}\NormalTok{()}
\ControlFlowTok{for}\NormalTok{(i }\ControlFlowTok{in} \DecValTok{1}\OperatorTok{:}\KeywordTok{nrow}\NormalTok{(eval2))\{}
  \ControlFlowTok{for}\NormalTok{(j }\ControlFlowTok{in} \DecValTok{1}\OperatorTok{:}\KeywordTok{nrow}\NormalTok{(train2))\{}
\NormalTok{    aux[j,i]<-}\KeywordTok{dist}\NormalTok{(}\KeywordTok{rbind}\NormalTok{(train2[j,}\OperatorTok{-}\KeywordTok{ncol}\NormalTok{(train2)], eval2[i,}\OperatorTok{-}\KeywordTok{ncol}\NormalTok{(eval2)]), }\DataTypeTok{method=}\StringTok{"maximum"}\NormalTok{) }
\NormalTok{  \}}
\NormalTok{\}}
\NormalTok{max<-aux}
\NormalTok{max}
\end{Highlighting}
\end{Shaded}

\begin{verbatim}
##           V1         V2        V3     V4
## 1  0.5000000 0.50000000 1.0000000 1.0000
## 2  1.0000000 1.00000000 0.6842105 1.0000
## 3  0.2631579 0.04878049 1.0000000 1.0000
## 4  0.3414634 0.17073171 1.0000000 1.0000
## 5  0.5000000 0.50000000 1.0000000 1.0000
## 6  1.0000000 1.00000000 1.0000000 1.0000
## 7  1.0000000 1.00000000 1.0000000 0.5000
## 8  1.0000000 1.00000000 1.0000000 0.1875
## 9  1.0000000 1.00000000 0.3500000 1.0000
## 10 0.2631579 0.07317073 1.0000000 1.0000
## 11 0.5000000 0.50000000 1.0000000 1.0000
## 12 0.1250000 0.26315789 1.0000000 1.0000
\end{verbatim}

Se creó la función para tomar el promedio del pronóstico de los
k-vecinos más cercanos de cada punto de evaluación:

\begin{Shaded}
\begin{Highlighting}[]
\NormalTok{kesimo<-}\ControlFlowTok{function}\NormalTok{(v,k)\{}
  \KeywordTok{mean}\NormalTok{(train2}\OperatorTok{$}\NormalTok{Invertir[}\KeywordTok{which}\NormalTok{(v }\OperatorTok\StringTok{ }\KeywordTok{sort}\NormalTok{(v)[}\DecValTok{1}\OperatorTok{:}\NormalTok{k])])}
\NormalTok{\}}
\end{Highlighting}
\end{Shaded}

Se aplicó a cada punto de evaluación para k=1,3 y 5 para la distancia
euclidiana quedando:

\begin{Shaded}
\begin{Highlighting}[]
\NormalTok{matriz_inversion_e<-}\KeywordTok{data.frame}\NormalTok{()}
\ControlFlowTok{for}\NormalTok{(v }\ControlFlowTok{in} \DecValTok{1}\OperatorTok{:}\KeywordTok{nrow}\NormalTok{(eval2))\{}
  \ControlFlowTok{for}\NormalTok{(k }\ControlFlowTok{in} \DecValTok{1}\OperatorTok{:}\DecValTok{3}\NormalTok{)\{}
\NormalTok{    matriz_inversion_e[v,k]<-}\KeywordTok{kesimo}\NormalTok{(euclidian[,v],k}\OperatorTok{*}\DecValTok{2}\OperatorTok{-}\DecValTok{1}\NormalTok{)}
\NormalTok{  \}}
\NormalTok{\}}
\KeywordTok{colnames}\NormalTok{(matriz_inversion_e)<-}\KeywordTok{c}\NormalTok{(}\StringTok{"k=1"}\NormalTok{,}\StringTok{"k=3"}\NormalTok{,}\StringTok{"k=5"}\NormalTok{)}
\NormalTok{matriz_inversion_e}
\end{Highlighting}
\end{Shaded}

\begin{verbatim}
##   k=1       k=3 k=5
## 1   1 1.0000000 1.0
## 2   1 1.0000000 0.8
## 3   1 0.6666667 0.8
## 4   1 0.6666667 0.6
\end{verbatim}

De donde se puede observar que para k=1 sugiere invertir en todos los
casos, mientras que para k=3 y 5; se puede definir un parámetro de
``confianza'', si se tomara .5 (redondeo) se invertiría en todos los
casos, si se tomara un parámetro de confianza más alto, tal vez
quedarían fuera las inversiones en los casos 3 y 4 (e incluso 2).

Para la distancia máxima se tiene:

\begin{Shaded}
\begin{Highlighting}[]
\NormalTok{matriz_inversion_max<-}\KeywordTok{data.frame}\NormalTok{()}
\ControlFlowTok{for}\NormalTok{(v }\ControlFlowTok{in} \DecValTok{1}\OperatorTok{:}\KeywordTok{nrow}\NormalTok{(eval2))\{}
  \ControlFlowTok{for}\NormalTok{(k }\ControlFlowTok{in} \DecValTok{1}\OperatorTok{:}\DecValTok{3}\NormalTok{)\{}
\NormalTok{    matriz_inversion_max[v,k]<-}\KeywordTok{kesimo}\NormalTok{(max[,v],k}\OperatorTok{*}\DecValTok{2}\OperatorTok{-}\DecValTok{1}\NormalTok{)}
\NormalTok{  \}}
\NormalTok{\}}
\KeywordTok{colnames}\NormalTok{(matriz_inversion_max)<-}\KeywordTok{c}\NormalTok{(}\StringTok{"k=1"}\NormalTok{,}\StringTok{"k=3"}\NormalTok{,}\StringTok{"k=5"}\NormalTok{)}
\NormalTok{matriz_inversion_max}
\end{Highlighting}
\end{Shaded}

\begin{verbatim}
##   k=1       k=3       k=5
## 1   1 1.0000000 0.7142857
## 2   1 1.0000000 0.7142857
## 3   1 0.5833333 0.5833333
## 4   1 0.5833333 0.5833333
\end{verbatim}

En donde para k=1 se sugiere invertir en todos los casos y para k=3 y 5
podría variar de acuerdo a un parámetro de ``confianza''.


\end{document}
