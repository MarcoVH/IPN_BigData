\documentclass[]{article}
\usepackage{lmodern}
\usepackage{amssymb,amsmath}
\usepackage{ifxetex,ifluatex}
\usepackage{fixltx2e} % provides \textsubscript
\ifnum 0\ifxetex 1\fi\ifluatex 1\fi=0 % if pdftex
  \usepackage[T1]{fontenc}
  \usepackage[utf8]{inputenc}
\else % if luatex or xelatex
  \ifxetex
    \usepackage{mathspec}
  \else
    \usepackage{fontspec}
  \fi
  \defaultfontfeatures{Ligatures=TeX,Scale=MatchLowercase}
\fi
% use upquote if available, for straight quotes in verbatim environments
\IfFileExists{upquote.sty}{\usepackage{upquote}}{}
% use microtype if available
\IfFileExists{microtype.sty}{%
\usepackage{microtype}
\UseMicrotypeSet[protrusion]{basicmath} % disable protrusion for tt fonts
}{}
\usepackage[margin=1in]{geometry}
\usepackage{hyperref}
\hypersetup{unicode=true,
            pdfborder={0 0 0},
            breaklinks=true}
\urlstyle{same}  % don't use monospace font for urls
\usepackage{color}
\usepackage{fancyvrb}
\newcommand{\VerbBar}{|}
\newcommand{\VERB}{\Verb[commandchars=\\\{\}]}
\DefineVerbatimEnvironment{Highlighting}{Verbatim}{commandchars=\\\{\}}
% Add ',fontsize=\small' for more characters per line
\usepackage{framed}
\definecolor{shadecolor}{RGB}{248,248,248}
\newenvironment{Shaded}{\begin{snugshade}}{\end{snugshade}}
\newcommand{\KeywordTok}[1]{\textcolor[rgb]{0.13,0.29,0.53}{\textbf{#1}}}
\newcommand{\DataTypeTok}[1]{\textcolor[rgb]{0.13,0.29,0.53}{#1}}
\newcommand{\DecValTok}[1]{\textcolor[rgb]{0.00,0.00,0.81}{#1}}
\newcommand{\BaseNTok}[1]{\textcolor[rgb]{0.00,0.00,0.81}{#1}}
\newcommand{\FloatTok}[1]{\textcolor[rgb]{0.00,0.00,0.81}{#1}}
\newcommand{\ConstantTok}[1]{\textcolor[rgb]{0.00,0.00,0.00}{#1}}
\newcommand{\CharTok}[1]{\textcolor[rgb]{0.31,0.60,0.02}{#1}}
\newcommand{\SpecialCharTok}[1]{\textcolor[rgb]{0.00,0.00,0.00}{#1}}
\newcommand{\StringTok}[1]{\textcolor[rgb]{0.31,0.60,0.02}{#1}}
\newcommand{\VerbatimStringTok}[1]{\textcolor[rgb]{0.31,0.60,0.02}{#1}}
\newcommand{\SpecialStringTok}[1]{\textcolor[rgb]{0.31,0.60,0.02}{#1}}
\newcommand{\ImportTok}[1]{#1}
\newcommand{\CommentTok}[1]{\textcolor[rgb]{0.56,0.35,0.01}{\textit{#1}}}
\newcommand{\DocumentationTok}[1]{\textcolor[rgb]{0.56,0.35,0.01}{\textbf{\textit{#1}}}}
\newcommand{\AnnotationTok}[1]{\textcolor[rgb]{0.56,0.35,0.01}{\textbf{\textit{#1}}}}
\newcommand{\CommentVarTok}[1]{\textcolor[rgb]{0.56,0.35,0.01}{\textbf{\textit{#1}}}}
\newcommand{\OtherTok}[1]{\textcolor[rgb]{0.56,0.35,0.01}{#1}}
\newcommand{\FunctionTok}[1]{\textcolor[rgb]{0.00,0.00,0.00}{#1}}
\newcommand{\VariableTok}[1]{\textcolor[rgb]{0.00,0.00,0.00}{#1}}
\newcommand{\ControlFlowTok}[1]{\textcolor[rgb]{0.13,0.29,0.53}{\textbf{#1}}}
\newcommand{\OperatorTok}[1]{\textcolor[rgb]{0.81,0.36,0.00}{\textbf{#1}}}
\newcommand{\BuiltInTok}[1]{#1}
\newcommand{\ExtensionTok}[1]{#1}
\newcommand{\PreprocessorTok}[1]{\textcolor[rgb]{0.56,0.35,0.01}{\textit{#1}}}
\newcommand{\AttributeTok}[1]{\textcolor[rgb]{0.77,0.63,0.00}{#1}}
\newcommand{\RegionMarkerTok}[1]{#1}
\newcommand{\InformationTok}[1]{\textcolor[rgb]{0.56,0.35,0.01}{\textbf{\textit{#1}}}}
\newcommand{\WarningTok}[1]{\textcolor[rgb]{0.56,0.35,0.01}{\textbf{\textit{#1}}}}
\newcommand{\AlertTok}[1]{\textcolor[rgb]{0.94,0.16,0.16}{#1}}
\newcommand{\ErrorTok}[1]{\textcolor[rgb]{0.64,0.00,0.00}{\textbf{#1}}}
\newcommand{\NormalTok}[1]{#1}
\usepackage{graphicx,grffile}
\makeatletter
\def\maxwidth{\ifdim\Gin@nat@width>\linewidth\linewidth\else\Gin@nat@width\fi}
\def\maxheight{\ifdim\Gin@nat@height>\textheight\textheight\else\Gin@nat@height\fi}
\makeatother
% Scale images if necessary, so that they will not overflow the page
% margins by default, and it is still possible to overwrite the defaults
% using explicit options in \includegraphics[width, height, ...]{}
\setkeys{Gin}{width=\maxwidth,height=\maxheight,keepaspectratio}
\IfFileExists{parskip.sty}{%
\usepackage{parskip}
}{% else
\setlength{\parindent}{0pt}
\setlength{\parskip}{6pt plus 2pt minus 1pt}
}
\setlength{\emergencystretch}{3em}  % prevent overfull lines
\providecommand{\tightlist}{%
  \setlength{\itemsep}{0pt}\setlength{\parskip}{0pt}}
\setcounter{secnumdepth}{0}
% Redefines (sub)paragraphs to behave more like sections
\ifx\paragraph\undefined\else
\let\oldparagraph\paragraph
\renewcommand{\paragraph}[1]{\oldparagraph{#1}\mbox{}}
\fi
\ifx\subparagraph\undefined\else
\let\oldsubparagraph\subparagraph
\renewcommand{\subparagraph}[1]{\oldsubparagraph{#1}\mbox{}}
\fi

%%% Use protect on footnotes to avoid problems with footnotes in titles
\let\rmarkdownfootnote\footnote%
\def\footnote{\protect\rmarkdownfootnote}

%%% Change title format to be more compact
\usepackage{titling}

% Create subtitle command for use in maketitle
\newcommand{\subtitle}[1]{
  \posttitle{
    \begin{center}\large#1\end{center}
    }
}

\setlength{\droptitle}{-2em}

  \title{}
    \pretitle{\vspace{\droptitle}}
  \posttitle{}
    \author{}
    \preauthor{}\postauthor{}
    \date{}
    \predate{}\postdate{}
  
\usepackage{float}

\begin{document}

\begin{centering}

\vspace*{5 cm}

\Huge

{\bf Descubrimiento del Conocimiento usando herramientas de Big Data Módulo 2}

\vspace{3 cm}

\Large
Marco Andrés Vázquez Hernández

\vspace{1 cm}
\normalsize
Práctica Patrones Comúnes. 

Septiembre de 2018

\normalsize
Instituto Politécnico Nacional


\end{centering}

\newpage

\section{Descripción}\label{descripcion}

Utilizar los datos que se proporcionan para encontrar:

1.- Representación de las transacciones de cada mes (Transacción,
{[}Productos{]})

2.- Representación binaria de los datos de transacciones

3.- Encontrar los patrones frecuentes utilizando un soporte de 0.001

\section{Planteamiento}\label{planteamiento}

Se plantea la pregunta a contestar: ¿Que secciones de la tienda deberían
de estar juntas (físicamente) para pomover las ventas?

Se tomó la variable product\_subcategory como indicadora de las
secciones de la tienda. como dicha variable está contenida en los datos
de las ventas mensuales, no se usaron los datos del archivo de productos

\section{Carga de archivos}\label{carga-de-archivos}

\begin{Shaded}
\begin{Highlighting}[]
\KeywordTok{library}\NormalTok{(}\StringTok{"arules"}\NormalTok{)}
\end{Highlighting}
\end{Shaded}

\begin{verbatim}
## Loading required package: Matrix
\end{verbatim}

\begin{verbatim}
## 
## Attaching package: 'arules'
\end{verbatim}

\begin{verbatim}
## The following objects are masked from 'package:base':
## 
##     abbreviate, write
\end{verbatim}

\begin{Shaded}
\begin{Highlighting}[]
\KeywordTok{library}\NormalTok{(}\StringTok{"ggplot2"}\NormalTok{)}
\CommentTok{#install.packages("magrittr")}
\KeywordTok{library}\NormalTok{(}\StringTok{"magrittr"}\NormalTok{)}
\KeywordTok{setwd}\NormalTok{(}\StringTok{"C:/Users/marco/IPN_BigData/Modulo2/Practica_patrones_frecuentes"}\NormalTok{)}

\NormalTok{ventas01<-}\KeywordTok{read.table}\NormalTok{(}\StringTok{"sales_01_Jan"}\NormalTok{,}\DataTypeTok{nrows =} \OperatorTok{-}\DecValTok{1}\NormalTok{,}\DataTypeTok{sep=}\StringTok{"|"}\NormalTok{,}\DataTypeTok{quote=}\StringTok{"}\CharTok{\textbackslash{}"}\StringTok{"}\NormalTok{,}\DataTypeTok{header=}\NormalTok{T)}
\NormalTok{ventas02<-}\KeywordTok{read.table}\NormalTok{(}\StringTok{"sales_02_Feb"}\NormalTok{,}\DataTypeTok{nrows =} \OperatorTok{-}\DecValTok{1}\NormalTok{,}\DataTypeTok{sep=}\StringTok{"|"}\NormalTok{,}\DataTypeTok{quote=}\StringTok{"}\CharTok{\textbackslash{}"}\StringTok{"}\NormalTok{,}\DataTypeTok{header=}\NormalTok{T)}
\NormalTok{ventas03<-}\KeywordTok{read.table}\NormalTok{(}\StringTok{"sales_03_Mar"}\NormalTok{,}\DataTypeTok{nrows =} \OperatorTok{-}\DecValTok{1}\NormalTok{,}\DataTypeTok{sep=}\StringTok{"|"}\NormalTok{,}\DataTypeTok{quote=}\StringTok{"}\CharTok{\textbackslash{}"}\StringTok{"}\NormalTok{,}\DataTypeTok{header=}\NormalTok{T)}
\NormalTok{ventas04<-}\KeywordTok{read.table}\NormalTok{(}\StringTok{"sales_04_Apr"}\NormalTok{,}\DataTypeTok{nrows =} \OperatorTok{-}\DecValTok{1}\NormalTok{,}\DataTypeTok{sep=}\StringTok{"|"}\NormalTok{,}\DataTypeTok{quote=}\StringTok{"}\CharTok{\textbackslash{}"}\StringTok{"}\NormalTok{,}\DataTypeTok{header=}\NormalTok{T)}
\NormalTok{ventas05<-}\KeywordTok{read.table}\NormalTok{(}\StringTok{"sales_05_May"}\NormalTok{,}\DataTypeTok{nrows =} \OperatorTok{-}\DecValTok{1}\NormalTok{,}\DataTypeTok{sep=}\StringTok{"|"}\NormalTok{,}\DataTypeTok{quote=}\StringTok{"}\CharTok{\textbackslash{}"}\StringTok{"}\NormalTok{,}\DataTypeTok{header=}\NormalTok{T)}
\NormalTok{ventas06<-}\KeywordTok{read.table}\NormalTok{(}\StringTok{"sales_06_Jun"}\NormalTok{,}\DataTypeTok{nrows =} \OperatorTok{-}\DecValTok{1}\NormalTok{,}\DataTypeTok{sep=}\StringTok{"|"}\NormalTok{,}\DataTypeTok{quote=}\StringTok{"}\CharTok{\textbackslash{}"}\StringTok{"}\NormalTok{,}\DataTypeTok{header=}\NormalTok{T)}
\NormalTok{ventas07<-}\KeywordTok{read.table}\NormalTok{(}\StringTok{"sales_07_Jul"}\NormalTok{,}\DataTypeTok{nrows =} \OperatorTok{-}\DecValTok{1}\NormalTok{,}\DataTypeTok{sep=}\StringTok{"|"}\NormalTok{,}\DataTypeTok{quote=}\StringTok{"}\CharTok{\textbackslash{}"}\StringTok{"}\NormalTok{,}\DataTypeTok{header=}\NormalTok{T)}
\NormalTok{ventas08<-}\KeywordTok{read.table}\NormalTok{(}\StringTok{"sales_08_Aug"}\NormalTok{,}\DataTypeTok{nrows =} \OperatorTok{-}\DecValTok{1}\NormalTok{,}\DataTypeTok{sep=}\StringTok{"|"}\NormalTok{,}\DataTypeTok{quote=}\StringTok{"}\CharTok{\textbackslash{}"}\StringTok{"}\NormalTok{,}\DataTypeTok{header=}\NormalTok{T)}
\NormalTok{ventas09<-}\KeywordTok{read.table}\NormalTok{(}\StringTok{"sales_09_Sep"}\NormalTok{,}\DataTypeTok{nrows =} \OperatorTok{-}\DecValTok{1}\NormalTok{,}\DataTypeTok{sep=}\StringTok{"|"}\NormalTok{,}\DataTypeTok{quote=}\StringTok{"}\CharTok{\textbackslash{}"}\StringTok{"}\NormalTok{,}\DataTypeTok{header=}\NormalTok{T)}
\NormalTok{ventas10<-}\KeywordTok{read.table}\NormalTok{(}\StringTok{"sales_10_Oct"}\NormalTok{,}\DataTypeTok{nrows =} \OperatorTok{-}\DecValTok{1}\NormalTok{,}\DataTypeTok{sep=}\StringTok{"|"}\NormalTok{,}\DataTypeTok{quote=}\StringTok{"}\CharTok{\textbackslash{}"}\StringTok{"}\NormalTok{,}\DataTypeTok{header=}\NormalTok{T)}
\NormalTok{ventas11<-}\KeywordTok{read.table}\NormalTok{(}\StringTok{"sales_11_Nov"}\NormalTok{,}\DataTypeTok{nrows =} \OperatorTok{-}\DecValTok{1}\NormalTok{,}\DataTypeTok{sep=}\StringTok{"|"}\NormalTok{,}\DataTypeTok{quote=}\StringTok{"}\CharTok{\textbackslash{}"}\StringTok{"}\NormalTok{,}\DataTypeTok{header=}\NormalTok{T)}
\NormalTok{ventas12<-}\KeywordTok{read.table}\NormalTok{(}\StringTok{"sales_12_Dec"}\NormalTok{,}\DataTypeTok{nrows =} \OperatorTok{-}\DecValTok{1}\NormalTok{,}\DataTypeTok{sep=}\StringTok{"|"}\NormalTok{,}\DataTypeTok{quote=}\StringTok{"}\CharTok{\textbackslash{}"}\StringTok{"}\NormalTok{,}\DataTypeTok{header=}\NormalTok{T)}
\NormalTok{ventas<-}\KeywordTok{rbind}\NormalTok{(ventas01,ventas02,ventas03,ventas04,ventas05,ventas06,ventas07,}
\NormalTok{              ventas08,ventas09,ventas10,ventas11,}
\NormalTok{              ventas12)}
\end{Highlighting}
\end{Shaded}

\section{Transformación de datos}\label{transformacion-de-datos}

Para usar la librería de R llamada ``arules'' que hace uso del algoritmo
apriori para detección de patrones frecuentes se deben de transformar
los datos a una estructura llave - valor y después guardarlos en un csv
y leerlos con la función read.transactions

\begin{Shaded}
\begin{Highlighting}[]
\NormalTok{aux<-ventas[,}\KeywordTok{c}\NormalTok{(}\StringTok{"time_id"}\NormalTok{,}\StringTok{"customer_id"}\NormalTok{,}\StringTok{"product_subcategory"}\NormalTok{)]}
\NormalTok{aux}\OperatorTok{$}\NormalTok{llave<-}\KeywordTok{paste0}\NormalTok{(aux}\OperatorTok{$}\NormalTok{time_id,}\StringTok{"-"}\NormalTok{,aux}\OperatorTok{$}\NormalTok{customer_id)}
\NormalTok{aux}\OperatorTok{$}\NormalTok{time_id<-}\OtherTok{NULL}
\NormalTok{aux}\OperatorTok{$}\NormalTok{customer_id<-}\OtherTok{NULL}
\NormalTok{aux<-aux[}\OperatorTok{!}\KeywordTok{duplicated}\NormalTok{(aux),]}
\KeywordTok{write.table}\NormalTok{(aux[,}\KeywordTok{c}\NormalTok{(}\DecValTok{2}\NormalTok{,}\DecValTok{1}\NormalTok{)],}\StringTok{"transacciones.csv"}\NormalTok{,}\DataTypeTok{quote=}\NormalTok{F,}\DataTypeTok{row.names=}\NormalTok{F,}\DataTypeTok{sep=}\StringTok{","}\NormalTok{)}

\NormalTok{transacciones<-}\KeywordTok{read.transactions}\NormalTok{(}\StringTok{"transacciones.csv"}\NormalTok{,}\DataTypeTok{format=}\StringTok{"single"}\NormalTok{,}\DataTypeTok{sep=}\StringTok{","}\NormalTok{,}\DataTypeTok{cols =}
                                   \KeywordTok{c}\NormalTok{(}\StringTok{"llave"}\NormalTok{,}\StringTok{"product_subcategory"}\NormalTok{))}
\end{Highlighting}
\end{Shaded}

\section{Análisis y resultados}\label{analisis-y-resultados}

Se usaron dichos datos para evaluar el algoritmo apriori y obtener los
patrones más frecuentes. Se utilizó un soporte de 0.02 ya que se tienen
suficientes datos de transacciones y ``pocas'' categorías.

\begin{Shaded}
\begin{Highlighting}[]
\NormalTok{itemsets <-}\StringTok{ }\KeywordTok{apriori}\NormalTok{(}\DataTypeTok{data =}\NormalTok{ transacciones,}
                    \DataTypeTok{parameter =} \KeywordTok{list}\NormalTok{(}\DataTypeTok{support =} \FloatTok{0.02}\NormalTok{,}
                                     \DataTypeTok{minlen =} \DecValTok{2}\NormalTok{,}
                                     \DataTypeTok{maxlen =} \DecValTok{20}\NormalTok{,}
                                     \DataTypeTok{target =} \StringTok{"frequent itemset"}\NormalTok{))}
\end{Highlighting}
\end{Shaded}

\begin{verbatim}
## Apriori
## 
## Parameter specification:
##  confidence minval smax arem  aval originalSupport maxtime support minlen
##          NA    0.1    1 none FALSE            TRUE       5    0.02      2
##  maxlen            target   ext
##      20 frequent itemsets FALSE
## 
## Algorithmic control:
##  filter tree heap memopt load sort verbose
##     0.1 TRUE TRUE  FALSE TRUE    2    TRUE
## 
## Absolute minimum support count: 410 
## 
## set item appearances ...[0 item(s)] done [0.00s].
## set transactions ...[102 item(s), 20522 transaction(s)] done [0.00s].
## sorting and recoding items ... [67 item(s)] done [0.00s].
## creating transaction tree ... done [0.00s].
## checking subsets of size 1 2 3 done [0.00s].
## writing ... [10 set(s)] done [0.00s].
## creating S4 object  ... done [0.00s].
\end{verbatim}

\begin{Shaded}
\begin{Highlighting}[]
\NormalTok{order_itemsets <-}\StringTok{ }\KeywordTok{sort}\NormalTok{(itemsets, }\DataTypeTok{by =} \StringTok{"support"}\NormalTok{, }\DataTypeTok{decreasing =} \OtherTok{TRUE}\NormalTok{)}
\KeywordTok{inspect}\NormalTok{(order_itemsets)}
\end{Highlighting}
\end{Shaded}

\begin{verbatim}
##      items                                support    count
## [1]  {Fresh Fruit,Fresh Vegetables}       0.04804600 986  
## [2]  {Fresh Vegetables,Soup}              0.03284280 674  
## [3]  {Dried Fruit,Fresh Vegetables}       0.03250171 667  
## [4]  {Cheese,Fresh Vegetables}            0.03094240 635  
## [5]  {Cookies,Fresh Vegetables}           0.02723906 559  
## [6]  {Fresh Vegetables,Wine}              0.02173277 446  
## [7]  {Canned Vegetables,Fresh Vegetables} 0.02163532 444  
## [8]  {Fresh Vegetables,Paper Wipes}       0.02109931 433  
## [9]  {Cheese,Fresh Fruit}                 0.02070948 425  
## [10] {Fresh Fruit,Soup}                   0.02051457 421
\end{verbatim}

\section{Visualización}\label{visualizacion}

Se presenta la visualización de dichos patrones por medio de una gráfica
de barras:

\begin{Shaded}
\begin{Highlighting}[]
\KeywordTok{as}\NormalTok{(order_itemsets, }\DataTypeTok{Class =} \StringTok{"data.frame"}\NormalTok{) }\OperatorTok
\StringTok{  }\KeywordTok{ggplot}\NormalTok{(}\KeywordTok{aes}\NormalTok{(}\DataTypeTok{x =} \KeywordTok{reorder}\NormalTok{(items, support), }\DataTypeTok{y =}\NormalTok{ support)) }\OperatorTok{+}
\StringTok{  }\KeywordTok{geom_col}\NormalTok{() }\OperatorTok{+}
\StringTok{  }\KeywordTok{coord_flip}\NormalTok{() }\OperatorTok{+}
\StringTok{  }\KeywordTok{labs}\NormalTok{(}\DataTypeTok{title =} \StringTok{"Itemsets más frecuentes"}\NormalTok{, }\DataTypeTok{x =} \StringTok{"itemsets"}\NormalTok{) }\OperatorTok{+}
\StringTok{  }\KeywordTok{theme_bw}\NormalTok{()}
\end{Highlighting}
\end{Shaded}

\includegraphics{patrones_freq_files/figure-latex/unnamed-chunk-4-1.pdf}


\end{document}
