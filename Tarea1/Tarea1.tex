\documentclass[]{article}
\usepackage{lmodern}
\usepackage{amssymb,amsmath}
\usepackage{ifxetex,ifluatex}
\usepackage{fixltx2e} % provides \textsubscript
\ifnum 0\ifxetex 1\fi\ifluatex 1\fi=0 % if pdftex
  \usepackage[T1]{fontenc}
  \usepackage[utf8]{inputenc}
\else % if luatex or xelatex
  \ifxetex
    \usepackage{mathspec}
  \else
    \usepackage{fontspec}
  \fi
  \defaultfontfeatures{Ligatures=TeX,Scale=MatchLowercase}
\fi
% use upquote if available, for straight quotes in verbatim environments
\IfFileExists{upquote.sty}{\usepackage{upquote}}{}
% use microtype if available
\IfFileExists{microtype.sty}{%
\usepackage{microtype}
\UseMicrotypeSet[protrusion]{basicmath} % disable protrusion for tt fonts
}{}
\usepackage[margin=1in]{geometry}
\usepackage{hyperref}
\hypersetup{unicode=true,
            pdfborder={0 0 0},
            breaklinks=true}
\urlstyle{same}  % don't use monospace font for urls
\usepackage{graphicx,grffile}
\makeatletter
\def\maxwidth{\ifdim\Gin@nat@width>\linewidth\linewidth\else\Gin@nat@width\fi}
\def\maxheight{\ifdim\Gin@nat@height>\textheight\textheight\else\Gin@nat@height\fi}
\makeatother
% Scale images if necessary, so that they will not overflow the page
% margins by default, and it is still possible to overwrite the defaults
% using explicit options in \includegraphics[width, height, ...]{}
\setkeys{Gin}{width=\maxwidth,height=\maxheight,keepaspectratio}
\IfFileExists{parskip.sty}{%
\usepackage{parskip}
}{% else
\setlength{\parindent}{0pt}
\setlength{\parskip}{6pt plus 2pt minus 1pt}
}
\setlength{\emergencystretch}{3em}  % prevent overfull lines
\providecommand{\tightlist}{%
  \setlength{\itemsep}{0pt}\setlength{\parskip}{0pt}}
\setcounter{secnumdepth}{0}
% Redefines (sub)paragraphs to behave more like sections
\ifx\paragraph\undefined\else
\let\oldparagraph\paragraph
\renewcommand{\paragraph}[1]{\oldparagraph{#1}\mbox{}}
\fi
\ifx\subparagraph\undefined\else
\let\oldsubparagraph\subparagraph
\renewcommand{\subparagraph}[1]{\oldsubparagraph{#1}\mbox{}}
\fi

%%% Use protect on footnotes to avoid problems with footnotes in titles
\let\rmarkdownfootnote\footnote%
\def\footnote{\protect\rmarkdownfootnote}

%%% Change title format to be more compact
\usepackage{titling}

% Create subtitle command for use in maketitle
\newcommand{\subtitle}[1]{
  \posttitle{
    \begin{center}\large#1\end{center}
    }
}

\setlength{\droptitle}{-2em}

  \title{}
    \pretitle{\vspace{\droptitle}}
  \posttitle{}
    \author{}
    \preauthor{}\postauthor{}
    \date{}
    \predate{}\postdate{}
  
\usepackage{booktabs}
\usepackage{longtable}
\usepackage{array}
\usepackage{multirow}
\usepackage[table]{xcolor}
\usepackage{wrapfig}
\usepackage{float}
\usepackage{colortbl}
\usepackage{pdflscape}
\usepackage{tabu}
\usepackage{threeparttable}
\usepackage{threeparttablex}
\usepackage[normalem]{ulem}
\usepackage{makecell}

\usepackage{float}

\begin{document}

\begin{centering}

\vspace*{5 cm}

\Huge

{\bf Introducción a Python}

\vspace{3 cm}

\Large
Marco Andrés Vázquez Hernández

\vspace{1 cm}
\normalsize
Tarea 1 

Agosto de 2018

\normalsize
Instituto Politécnico Nacional


\end{centering}

\newpage

\section{Descripción}\label{descripcion}

(Usando el archivo ``velocidades.csv'')\\
¿Cuántos registros tiene la base de datos?\\
¿Cuántos vehículos existen en la base de datos?\\
¿Cuantos dias aparecen, y de que fecha a que fecha son los registros?\\
¿Cuántos meses completos de información se tienen?\\
¿En qué horario trabaja la flota?\\
¿Cuál es la velocidad máxima registrada y que vehículo es?\\
El límite de velocidad máximo permitido es de 80 Km/h, ¿Cuántos
vehículos lo rebasan y cuales son?\\
¿Cuál es la hora con mayor frecuencia de excesos velocidad?\\
Tomando en cuenta los meses completos, ¿Cuál es la velocidad promedio de
cada mes?

\section{Reportes}\label{reportes}

\begin{table}[H]
\centering
\resizebox{\linewidth}{!}{
\begin{tabular}{rll}
\toprule
X & Concepto & Valor\\
\midrule
0 & No. de registros & 3588920\\
1 & Vehículos diferentes & 25\\
2 & No. de días diferentes & 45\\
3 & Rango de Fechas & De 2018-01-16 a 2018-04-04\\
4 & Meses completos & 2\\
\addlinespace
5 & Horario de la flota & De 00:00:00 a 23:59:59\\
6 & Velocidad máxima registrada & 92\\
7 & Vehículos que registran la velocidad máxima & ['b21', 'b19', 'b49', 'b24']\\
8 & No. de Vehículos que rebasan el límite de velocidad & 16\\
9 & Vehículos que rebasan el límite de velocidad & ['b2E', 'b17', 'b23', 'b46', 'b21', 'b16', 'b44', 'b4E', 'b2D', 'b17C', 'b19', 'b47', 'b4C', 'b2C', 'b49', 'b24']\\
10 & Horas con mayor frecuencia de excesos de velocidad & [11, 12]\\
\bottomrule
\end{tabular}}
\end{table}

Tomando en cuenta los meses completos, ¿Cuál es la velocidad promedio de
cada mes?

\begin{tabular}{lr}
\toprule
anomes & velocidad\\
\midrule
2018-2 & 14.92753\\
2018-3 & 15.61394\\
\bottomrule
\end{tabular}


\end{document}
